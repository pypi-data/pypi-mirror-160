\documentclass[aps,twocolumn,nofootinbib,nopreprintnumbers,showpacs,linenumbers,floatfix,longbibliography,superscriptaddress,prd]{revtex4-1}
%\usepackage{aas_macros}

\usepackage{color}
\usepackage{xcolor}
\usepackage{calc}
\usepackage{mathtools,graphicx}
\usepackage{pifont}
\usepackage{bm}
\usepackage{microtype}
\usepackage{booktabs}
\usepackage{times}
\usepackage[varg]{txfonts}
\usepackage[colorlinks, pdfborder={0 0 0}]{hyperref}
\usepackage[utf8]{inputenc}
\usepackage[caption=false]{subfig}
\usepackage{paralist}
\usepackage[normalem]{ulem}
\usepackage{multirow}
\usepackage{etoolbox}
\definecolor{LinkColor}{rgb}{0.75, 0, 0}
\definecolor{CiteColor}{rgb}{0, 0.5, 0.5}
\definecolor{UrlColor}{rgb}{0, 0, 0.75}
\hypersetup{linkcolor=LinkColor}
\hypersetup{citecolor=CiteColor}
\hypersetup{urlcolor=UrlColor}
\maxdeadcycles=1000
\allowdisplaybreaks
\DeclareFontFamily{OT1}{pzc}{}
\DeclareFontShape{OT1}{pzc}{m}{it}{<-> s * [1.10] pzcmi7t}{}
\DeclareMathAlphabet{\mathpzc}{OT1}{pzc}{m}{it}

\AtBeginDocument{%
    \newwrite\bibnotes
    \def\bibnotesext{Notes.bib}
    \immediate\openout\bibnotes=\jobname\bibnotesext
    \immediate\write\bibnotes{@CONTROL{REVTEX41Control}}
    \immediate\write\bibnotes{@CONTROL{%
    apsrev41Control,author="08",editor="1",pages="0",title="0",year="1"}}
     \if@filesw
     \immediate\write\@auxout{\string\citation{apsrev41Control}}%
    \fi
}%



%% Enable/disable the full author list
\newtoggle{fullauthorlist}
\toggletrue{fullauthorlist}
%\togglefalse{fullauthorlist}

%% Enable/disable placing author list at the end
\newtoggle{endauthorlist}
\toggletrue{endauthorlist}
%\togglefalse{endauthorlist}


%%Macros

% COMMANDS FOR SPARING WRITING TIME

\newcommand{\linf}{\textsc{LALInference}}
\newcommand{\linfmcmc}{\textsc{LALInferenceMCMC}}
\newcommand{\linfnest}{\textsc{LALInferenceNest}}
\newcommand{\bw}{\textsc{BayesWave}}
\newcommand{\gstlal}{\textsc{GstLAL}}
\newcommand{\pycbc}{\textsc{PyCBC}}
\newcommand{\cwb}{\textsc{cWB}}
\newcommand{\mc}{\ensuremath{\mathcal{M}}}
\newcommand{\msun}{\ensuremath{\mathrm{M}_\odot}}
\newcommand{\beq}{\begin{equation}}
\newcommand{\eeq}{\end{equation}}

\DeclareMathOperator{\sign}{sign}

\newcommand{\Fermi}{\emph{Fermi}\xspace}
\newcommand{\Swift}{\emph{Swift}\xspace}
\newcommand{\Epeak}{\ensuremath{E_{\rm peak}}\xspace}
\newcommand{\Eiso}{\ensuremath{E_{\rm iso}}\xspace}
\newcommand{\Liso}{\ensuremath{L_{\rm iso}}\xspace}

\def\fr#1#2{{{#1} \over {#2}}}
\def\half{{\textstyle{1\over 2}}}
% \def\frac#1#2{{\textstyle{{#1}\over {#2}}}}
% \def\etal{{\it et al.}}
\def\sb{\overline{s}{}}
\def\re{{\rm Re}~}
\def\im{{\rm Im}~}

% \newcommand{\bea}{\begin{eqnarray}}
% \newcommand{\eea}{\end{eqnarray}}

\newcommand{\jay}[1]{\textcolor{blue}{[JAY: #1]}}
\newcommand{\oli}[1]{{\textcolor{olive}{[Olivier: #1]}}}
\newcommand{\mina}[1]{{\textcolor{orange}{[Min-A: #1]}}}

\newcommand{\issue}[1]{{\textcolor{red}{ #1}}}
\newcommand{\raven}[1]{{\textcolor{green}{ #1}}}



%% Macros specific to this manuscript  to be filled (according to a specific given event)

%Template of maccros related to a given association of GW and GRB event

%(Automatically generated)

\newcommand{\urlVOEvent}{https://gracedb-test.ligo.org/api/superevents/MS220525t/files/MS220525t-2-Preliminary.xml,0}

\newcommand{\GWcolla}{LIGO Scientific Collaboration and Virgo Collaboration}

\newcommand{\GWskyn}{BAYESTAR}

\newcommand{\GWskyurl}{https://gracedb-test.ligo.org/api/superevents/MS220525t/files/bayestar.fits.gz,1}

\newcommand{\Hanford}{LIGO Hanford Observatory (H1)}

\newcommand{\Ha}{H1}

\newcommand{\Livingston}{LIGO Livingston Observatory (L1)}

\newcommand{\Li}{L1}

\newcommand{\Virgo}{Virgo Observatory (V1)}

\newcommand{\Vi}{V1}

\newcommand{\detectors}{\Hanford~and \Livingston}

\newcommand{\detectorsabvr}{\Ha, \Li}

\newcommand{\GWtime}{2022-05-25 18:41:22 UTC (GPS time: 1337539301)}

\newcommand{\Pevents}{BNS (100.00 \%), Terrestrial (0.00 \%), NSBH (0.00 \%) or BBH (0.00 \%)} 

\newcommand{\etype}{binary neutron star (BNS) merger}

\newcommand{\ety}{BNS}

\newcommand{\GWevent}{\textsc{MS220525t}}

\newcommand{\distance}{$93$ Mpc}

\newcommand{\distmean}{{147}}

\newcommand{\distdeltamin}{{-54}}

\newcommand{\distdeltamax}{{+60}}

\newcommand{\pipesup}{GSTLAL \cite{messik:2017pr,sachdev:2019ax}}

\newcommand{\partners}{\raven{Fermi Satellite, GBM Instrument}}
%In what follows: placeholders not automatized yet

\newcommand{\Pvalues}{$P = 1- \raven{2.80 \times 10^{-2}}$}
\newcommand{\Ssigma}{\raven{1.9 }$\sigma$}
\newcommand{\GRBtime}{\issue{YYYY-MM-DD HH:MM:SS.mmm UTC (GPS time: X)}}
\newcommand{\tdelay}{$\Delta t_{\text{SGRB--GW}} = \raven{2.00} \pm \issue{0.14} $ s}
\newcommand{\dtErrorPos}{\issue{0.02}}
\newcommand{\dtErrorLVCpipe}{\issue{0.07}}
\newcommand{\dtErrorFermipipe}{\issue{0.05}}
\newcommand{\vu}{\raven{+2.29 \times 10^{-16} }}
\newcommand{\vl}{\raven{-8.93 \times 10^{-16} }}
\newcommand{\FSDratioU}{\raven{xx~}}
\newcommand{\FSDratioL}{\raven{xx~}}
\newcommand{\GRBevent}{\raven{\textsc{M226084}}}
\newcommand{\GRBeventTitle}{\textsc{M226084}}
\newcommand{\saau}{\raven{+2 \times 10^{-15} }}
\newcommand{\saal}{\raven{-6 \times 10^{-15} }}
\newcommand{\mTdelay}{\raven{2.00}}
\newcommand{\deltaTdelay}{\issue{0.14}}


%% Define who are the GRB partners
%\newcommand{\partners}{GRB partners}


\begin{document}

\title{Low-latency report on the speed of gravity from gravitational wave and gamma ray burst coincidental events}
%\title{Report on the speed of gravity from \GWevent and \GRBevent}

\iftoggle{endauthorlist}{
 %
 % Put the author list at the end of the document.
 % Save author, affiliation, and maketitle commands.
 %
 \let\mymaketitle\maketitle
 \let\myauthor\author
 \let\myaffiliation\affiliation
 \author{\GWcolla~and \partners}
}{
 %
 % Keep the author list on the initial title page.
 %
 \iftoggle{fullauthorlist}{
 \input{lvc_authors}
 }{
 \author{The \GWcolla~and \partners}
}
}


\date[\relax]{compiled \today}

 
\begin{abstract}
The low-latency pipeline RAVEN finds that the gravitational wave candidate \GWevent~ and gamma ray burst \GRBevent~ events have a probability of association of \Pvalues, corresponding to a \Ssigma~significance in Gaussian statistics.  Assuming that the two events share the same origin, we use the low latency computations of the 90\% C.L. lower bound luminosity distance of the gravitational source (\distance) from \GWskyn~and of the time delay between the two events (\tdelay) in order to give an estimate of the new constraint on the fractional speed difference between gravitational and electromagnetic waves. We also derive the respective constraints on violations of Lorentz invariance.
%We use the low latency high association probability (\Pvalues) between the events \GWevent and \GRBevent, and the respective low latency time delay between the two events(\tdelay), in order to constrain the fractional speed difference between gravitational waves and photons. \textcolor{red}{We report an improvement of the fractional speed difference between the speed of gravitational waves and photons by a factor \FSDratioL and \FSDratioU for the lower and upper bounds respectively.[omit? omit if no improvement?]} We also derive the respective constraints on violations of Lorentz invariance \textcolor{red}{, some of which are improved relative to existing constraints [omit? omit if no improvement?]}.
\end{abstract}

\maketitle

The seminal first observation of gravitational waves (GW) and photons from a single astrophysical event in \cite{abbott:2017bnsgrb} brought forth a host of new scientific results. Among these results was confirmation that GWs and gamma rays generated by a binary neutron star merger
are observed with little to no difference in their arrival times.
This result was expected based both on
the thought that electromagnetic and gravitational waves propagate at the same speed
and the notion that the intrinsic emission times are similar.
Assuming the latter led to a 10 order of magnitude improvement in our experimental knowledge 
of the speed of gravitational waves relative to the speed of light
and associated constraints on deviations from known physics. \cite{abbott:2017bnsgrb}

In this work,
we repeat the speed-related analysis of \citet{abbott:2017bnsgrb} for the events 
\GWevent~ and \GRBevent~ that have been observed with a temporal offset of \tdelay.
In doing so,
we use the conventions established in \cite{abbott:2017bnsgrb}.
The paragraphs to follow
summarize the important aspects of our analysis
and provide the updated results.
Further updates to these low-latency results may be made
as additional information about this event becomes available.

%-----------------------------------------------------------------------------------------------------
\section{\GWevent}


The \GWcolla~have identified a GW candidate \GWevent~during real-time processing of data from \pipesup~ in \detectors. Because the classification of the signal in order of descending probability is\footnote{In the present manuscript, we consider the following VOEvent: \href{\urlVOEvent}{\urlVOEvent}.}: \Pevents, in what follows we assume that the event is a \ety. Marginalized over the whole sky, the luminosity distance is $\distmean^\distdeltamax_\distdeltamin$ Mpc at the 90\% C.L\footnote{From \href{\GWskyurl}{\GWskyurl}.}. Because the bounds of the relative difference between the speed of gravitational and electromagnetic waves are weaker for small distances, in what follows we use the lower bound of the 90\% credible interval of the luminosity distance \distance. 


%-----------------------------------------------------------------------------------------------------
\section{\GRBeventTitle}

\issue{The Fermi Gamma-ray Burst Monitor (GBM) triggered onboard in respond to GRB YYMMDDX (trigger XXXXXXXXX / YYMMDDXXX) at YYYY-MM-DD HH:MM:SS.mmm UTC. 
%Using the duration-based classification method as in \cite{goldstein2017ordinary}, the probability that this burst belongs to the short class is XX.
The gamma-ray properties of this event are consistent with a sGRB origin}

\issue{The Fermi Gamma-ray Burst Monitor blind untargeted search for SGRB candidates detected GRB YYMMDDX at YYYY-MM-DD UTC. The burst duration as identified by the search pipeline is XX seconds, consistent with a SGRB origin. The temporal and spatial consistency with \GWevent~has elevated the candidate to a confirmed SGRB.}

\issue{The on-set of gamma-ray emission is defined as the beginning of the detection source interval. This differs from the method in \citet{abbott:2017bnsgrb}, but is generally a more conservative definition.}

%-----------------------------------------------------------------------------------------------------
\section{Probability of association}

We are able to approximately calculate the probability of association \(P(\mathcal{H}_{c})\) using the tools of the Rapid, on-source VOEvent Coincidence Monitor (RAVEN). Since a coincidence can only be either due to a real astrophysical event (\(\mathcal{H}_{c}\)) or random chance (\(\mathcal{H}_{r}\)), this means we can write the probability of the former by
\begin{equation}
P(\mathcal{H}_{c} | x_{GW}, x_{GRB}) = 1 - P(\mathcal{H}_{r} | x_{GW}, x_{GRB}) \label{eq:prob_assoc_og}
\end{equation}
where \(x_{GW}\) and \(x_{GRB}\) are the data for the GW and GRB experiments respectively. The probability of random association due to timing can be given from the Poisson distribution
\begin{equation}
P(\mathcal{H}_{r}) = R_{GRB} \Delta t \exp( - R_{GRB} \Delta t) \approx R_{GRB} \Delta t
\end{equation}
where \(R_{GRB}\) is the rate of GRBs and \(\Delta t\) is the total time in the coincidence window. We can take this approximation due to the relatively low rate of GRBs and tight coincidence window.
We can also include sky map information by dividing by the sky map overlap integral (\(\mathcal{I}_{\Omega}\)) \cite{ashton:2018} to get the approximate expression
\begin{equation}
P(\mathcal{H}_{r} | x_{GW}, x_{GRB}) \approx R_{GRB} \Delta t / \mathcal{I}_{\Omega} .
\end{equation}
Since the joint false alarm rate (\(FAR_c\)) calculated in RAVEN is 
\begin{equation}
FAR_c = FAR_{gw} R_{GRB} \Delta t / \mathcal{I}_{\Omega} 
\end{equation}
where \(FAR_{gw}\) is the false alarm rate of the GW candidate, we can simply write the probability of association from equation \eqref{eq:prob_assoc_og} as
\begin{equation}
P(\mathcal{H}_{c} | x_{GW}, x_{GRB}) = 1 - FAR_c / FAR_{gw}
\end{equation}
which in this case \Pvalues, corresponding to a \Ssigma~significance in Gaussian statistics.

%-----------------------------------------------------------------------------------------------------
\section{Speed of gravity versus light}

To attain the fractional speed difference (FSD)
between GWs and light,
we begin by considering the arrival times $t_{\rm a,EM}$ and $t_{\rm a,GW}$
and emission times $t_{\rm emit,EM}$ and $t_{\rm emit,GW}$ 
of the EM and peak GW signals respectively.
Given existing limits,
we proceed under the assumption
that difference in the arrival times 
$\Delta t_{\rm a} = t_{\rm a,EM} - t_{\rm a,GW}$
and
emission times $\Delta t_{\rm emit} = t_{\rm emit,EM} - t_{\rm emit,GW}$
of the EM and GW signals are small,
as is the difference in the group velocities $\Delta v = v_{\rm GW} - v_{\rm EM}$
of the GW $v_{\rm GW}$ and EM $v_{\rm EM}$ signals.
Under these conditions,
the fractional speed difference (FSD) between GWs and light can be expressed
\begin{equation}
\label{eq:relspeed}
\fr{\Delta v}{v_{\rm EM}^2} = \fr{1+z}{D_L} \Big (\Delta t_{\rm a} - (1+z) \Delta t_{\rm emit} \Big),
\end{equation}
in terms of the luminosity distance $D_L$ and redshift $z$.
In the limit $z \rightarrow 0$,
this formula reduces to one applied in Ref.\ \citet{abbott:2017bnsgrb}
where $z$ was negligibly small.
Where as in Ref.\ \citet{abbott:2017bnsgrb} the result was written in terms of the travel-time difference,
here we replace the travel-time difference with the more explicit 
notation appearing in parenthesis in Eq.\ (\ref{eq:relspeed}).
Redshift effects are incorporated by equating the comoving distance
traveled by each messenger
and solving for the (constant) group-velocity difference \cite{will1998bounding,jacob:2008,mirshekari:2012,nishizawa2014measuring,kostelecky:16}.

In \citet{abbott:2017bnsgrb} a straight-forward procedure 
was used to obtain a two-sided conservative constraint on the FSD.
Here we summarize the approach and apply it to \GWevent~ and \GRBevent.
First, we note that Eq.\ (\ref{eq:relspeed}) is less constraining for smaller
distances.  Hence we use the lower bound of the 90\% credible interval on luminosity distance
derived from the GW signal for the distance $D_L$ throughout the analysis.
To achieve an upper bound on $\Delta v$, we assume that the peak of the GW signal and
the first photons were emitted simultaneously ($\Delta t_{\rm emit}=0$), attributing the entire
observed time lag to faster travel by the GW signal.  The lower
bound on $\Delta v$ is attained by assuming that the two signals were emitted
at times differing by more than the observed time lag with a faster light
signal making up a part of the difference. As a conservative bound
relative to the few seconds of emission time delay predicted in many astrophysical models
of the binary coalescences that are expected to produce GRBs,
\cite{aDetObsScenario,Zhang_2019}
we assume $\Delta t_{\rm emit}= 10 {\rm s}$.  
We also note that the effect of intergalactic medium dispersion on
the gamma-ray photon speed is neglected as the associated propagation delay 
that could be expected is many orders of
magnitude smaller than our errors on $v_{\rm GW}$.
With the redshift now incorporated into Eq.\ (\ref{eq:relspeed}),
we additionally obtain the value of $z$
corresponding to $D_L$
using the Planck 2015 results \cite{ade:2015} --- using parameters from table 4, column `TT+lowP+lensing+ext'. This corresponds to $\Omega_M = 0.3065$, $\Omega_\Lambda = 0.6935$ and $H_0 = 67.90$ km s$^{-1}$ Mpc$^{-1}$.
With the observed values of $D_L=$\distance\
and $\Delta t_{\rm a}=$\tdelay\ for \GWevent~and \GRBevent,
the constraint on the
FSD is
\begin{equation}
\label{eq:speed_GR}
\vl \leq \frac{\Delta v}{v_{\rm EM}} \leq \vu.
\end{equation}
The error in timing comes from various uncertainties: the LVC low-latency pipelines and GBM pipeline timing accuracies are better than \dtErrorLVCpipe~and \dtErrorFermipipe~s respectively, while the maximum correction for light/gravity travel time between all the network is less than \dtErrorPos~s.
In some alternative models,
lags much longer than 10\,s are proposed 
\cite[e.g.,][]{ciolfi:15, rezzolla:15}, and the emission of photons
before the merger may be possible \cite{tsang:2012}.  Hence, for certain
exotic scenarios the time difference window could be extended, perhaps to ($-100$\,s,
$1000$\,s), yielding a 2 orders of magnitude broadening of the allowed
velocity range on both sides.  

%Though the results of this approach 
%depend on astrophysical models of emission times, 
%the conservative assumptions reviewed here allow multimessenger astronomy
%to provide dramatic improvements over prior indirect
%\cite{Kostelecky:2008ts} and direct \cite{cornish:17} constraints,
%which allowed for time differences of more than 1000 years,
%in clear contradiction to our interpretation of current multimessenger events.  
Disentangling the emission time
difference from the relative propagation time
should become possible as additional events of this kind are observed,
since only the propagation time is expected to depend on distance.

\begin{widetext}
\begin{center}
\begin{table}
\begin{tabular}{|l|l|c|c|c|c|}
\hline
GW Event &GRB event  &Association p-value  &  FSD (lower bound)&FSD (higher bound) & Reference   \\
\hline
 GW 170817&GRB 170817A  & $ P=1- 5.0\times 10^{-8}$ ($5.3~\sigma$)& $-3 \times 10^{-15}$ & $+7 \times 10^{-16} $ & \cite{abbott:2017bnsgrb}\\
\hline
 \GWevent& \GRBevent &  \Pvalues~(\Ssigma)& $\vl$ & $\vu$ & present work \\
\hline
\end{tabular}
\caption{Summary of previous bounds on the fractional speed difference (FSD) between the speed of gravity and the speed of light. Note that the new bound comes from low-latency calculations. A final result will appear in a forthcoming publication.}
\end{table}
\end{center}
\end{widetext}

%-----------------------------------------------------------------------------------------------------
\section{Lorentz invariance}



We now use the constraints on the relative speed of the two messengers
achieved here
to place constraints on violations of Lorentz invariance 
within the comprehensive effective field theory framework
of the gravitational Standard-Model Extension (SME)
\cite{colladay:98,kostelecky:04,tasson:14}.
Again, we briefly summarize the approach used in \cite{abbott:2017bnsgrb}
before applying it to the current event.
In the SME expansion,
the relative group velocity of GWs and EM waves is controlled by differences in coefficients
for Lorentz violation in the photon and gravitational sectors 
of the theory at each mass dimension $d$
\cite{kostelecky:16,kostelecky:09,kostelecky:08}.  
We consider the non-birefringent, non-dispersive limit at mass dimension
$d=4$ since our current techniques excel over other kinds of measurements in this case.  
Here the difference in group velocities for the two sectors is written
%
\beq%
\Delta v = - \sum_{\substack{\ell m\\\ell \leq 2}} Y_{\ell m} (\hat n)
\left(\half (-1)^{1+\ell} \sb^{(4)}_{\ell m} - c^{(4)}_{(I)\ell m}
\right),
\label{rv}
\eeq%
where $Y_{\ell m}$ are the spherical harmonics.
The direction $\hat n$
refers to the sky direction of the event,
and the spherical-basis
coefficients for Lorentz violation in the gravitational and
photon sectors, respectively are $\sb^{(4)}_{\ell m}$ and $c^{(4)}_{(I)\ell m}$.

For ease of comparison with the many existing sensitivities
\cite{Kostelecky:2008ts} to the $d=4$ gravity-sector
coefficients, an analysis in which the coefficients
are constrained one at a time was used in Ref.\ \cite{abbott:2017bnsgrb}, with all
other coefficients, including those in the photon sector, set to zero.
This approach can be repeated when sufficient information about the sky position of the event is available,
such as from an optical counter part.
When sufficient information is not yet available,
we can still constrain the isotropic $\sb^{(4)}_{00}$. 
The results of this procedure for the current event are presented in Table \ref{tab:sme_coeff} along
with the best constraints for each coefficient prior to 
the observation of \GWevent~ and \GRBevent. 

%For ease of comparison with the many existing sensitivities
%\cite{Kostelecky:2008ts} to the $d=4$ gravity-sector
%coefficients, an analysis in which the coefficients
%are constrained one at a time is used, with all
%other coefficients, including those in the photon sector, set to zero.
%The results of this procedure for the current event are presented in Table \ref{tab:sme_coeff} along
%with the best constraints for each coefficient prior to 
%the observation of \GWevent and \GRBevent. \oli{I don't think we should give the coeff beyond the monopole because we won't have an accurate localization at this stage. Or it should involve new computations of credible intervals that have not bee reviewed.}

We note in passing that 
these Lorentz violation results can also be interpreted within the isotropic $A$, $\alpha_{\rm
  LV}$ Lorentz violation parametrization \cite{mirshekari:2012} used
by \citet{GW170104} in dispersive GW tests.
The isotropic, $d=4$ limit of the SME can be identified with the
$\alpha_{\rm LV} = 2$ limit of the $A$, $\alpha_{\rm
  LV}$ parametrization with
$\sb^{(4)}_{00} \rightarrow \sqrt{4 \pi} A$, and constraints on $A$ for
$\alpha_{\rm LV} = 2$ can be obtained from the first line of Table
\ref{tab:sme_coeff}.

\begin{widetext}
\begin{center}
\begin{table}
\label{tab:sme_coeff}
\begin{tabular}{|l|l|c|c|c|c|c|c|}
    \hline
      $\ell$	&	 Previous		&	 This Work	&	 Coefficient	&	 This Work	&	 Previous		\\
     \hline
	&	 Lower		&	 Lower	&		&	 Upper	&	 Upper		\\
0	&	 $-2 \times 10^{-14}$ 	\cite{abbott:2017bnsgrb} 	&	$\saal$	&	 $\sb^{(4)}_{00}$      	&	$\saau$	&	 $5 \times 10^{-15}$ 	\cite{abbott:2017bnsgrb} 	\\
    \hline													
\end{tabular}
\caption{Constraints on the dimensionless SME coefficient $\sb^{(4)}_{00}$ of the minimal gravity
      sector. The coefficient shown is constrained by setting all other
      coefficients, including those from the photon sector, to zero. }
\end{table}
\end{center}
\end{widetext}

%\begin{widetext}
%\begin{center}
%\begin{table}
%\label{tab:sme_coeff}
%\begin{tabular}{|l|l|c|c|c|c|c|c|}
%    \hline
%      $\ell$	&	 Previous		&	 This Work	&	 Coefficient	&	 This Work	&	 Previous		\\
%     \hline
%	&	 Lower		&	 Lower	&		&	 Upper	&	 Upper		\\
%0	&	 $-2 \times 10^{-14}$ 	\cite{abbott:2017bnsgrb} 	&	\saal	&	 $\sb^{(4)}_{00}$      	&	\saau	&	 $5 \times 10^{-15}$ 	\cite{abbott:2017bnsgrb} 	\\
%    \hline													
%1	&	 $-3 \times 10^{-14}$ 	\cite{abbott:2017bnsgrb} 	&	\sbal	&	 $ \sb^{(4)}_{10}$     	&	\sbau	&	 $7 \times 10^{-15}$ 	\cite{abbott:2017bnsgrb} 	\\
%      	&	 $-2 \times 10^{-15}$ 	\cite{abbott:2017bnsgrb} 	&	\resbbl	&	 $ \re \sb^{(4)}_{11}$           	&	\resbbu	&	 $1 \times 10^{-14}$ 	\cite{abbott:2017bnsgrb} 	\\
%      	&	 $-3 \times 10^{-14}$ 	\cite{abbott:2017bnsgrb} 	&	\imsbbl	&	 $ \im \sb^{(4)}_{11}$ 	&	\imsbbu	&	 $7 \times 10^{-15}$ 	\cite{abbott:2017bnsgrb} 	\\
%    \hline													
%2	&	 $-8 \times 10^{-15}$ 	\cite{abbott:2017bnsgrb} 	&	\scal	&	 $ \sb^{(4)}_{20}$               	&	\scau	&	 $4 \times 10^{-14}$ 	\cite{abbott:2017bnsgrb} 	\\
%      	&	 $-2 \times 10^{-15}$ 	\cite{abbott:2017bnsgrb} 	&	\rescbl	&	 $\re \sb^{(4)}_{21}$          	&	\rescbu	&	 $1 \times 10^{-14}$ 	\cite{abbott:2017bnsgrb} 	\\
%      	&	 $-4 \times 10^{-14}$ 	\cite{abbott:2017bnsgrb} 	&	\imscbl	&	 $ \im \sb^{(4)}_{21}$ 	&	\imscbu	&	 $8 \times 10^{-15}$ 	\cite{abbott:2017bnsgrb} 	\\
%      	&	 $-1 \times 10^{-14}$ 	\cite{abbott:2017bnsgrb} 	&	\resccl	&	 $ \re \sb^{(4)}_{22}$ 	&	\resccu	&	 $3 \times 10^{-15}$ 	\cite{abbott:2017bnsgrb} 	\\
%      	&	 $-4 \times 10^{-15}$ 	\cite{abbott:2017bnsgrb} 	&	\imsccl	&	 $\im \sb^{(4)}_{22}$           	&	\imsccu	&	 $2 \times 10^{-14}$ 	\cite{abbott:2017bnsgrb} 	\\
%\hline
%\end{tabular}
%\caption{Constraints on the dimensionless SME coefficients of the minimal gravity
%      sector. Columns 3 and 5 show constraints on the dimensionless SME coefficients of the minimal gravity
%      sector attained via Eqs.\,(\ref{eq:speed_GR}) and (\ref{rv}) using data form \GWevent\ and \GRBevent.
%      Columns 2 and 6 show the corresponding best limits achieved in the earlier work cited.
%      The coefficients shown are constrained one at a time, by setting all other
%      coefficients, including the others shown here and those from the photon sector, to zero. 
%      A dash, if present, indicates a coefficient not yet constrained by analysis of \GWevent and \GRBevent.
%}
%\end{table}
%\end{center}
%\end{widetext}



%%-----------------------------------------------------------------------------------------------------
%\section{Conclusion} 
%Low latency, therefore preliminary






\bibliography{SoG_LIV}
\end{document}

